\chapter{\abstractname}

Recent trends in the automotive industry are introducing new, increasingly complex and software-driven functions into vehicles. These functions put high demands on the the vehicle's on-board system. Autonomous driving and big data applications are the driving forces behind this development.

Context
\begin{itemize}
\item Vehicular functions become increasingly complex
\item Hardware in vehicles is expensive and constrained
\item Sensors are producing increased quantities of data and the vehicle intrinsic network needs to deal with increased bandwidth demands
\item There is a need for new architectures and technologies
\item cloud computing has long been established as a viable alternative to on-premise infrastructures. Infrastructure can be outsourced. In addition to cost reductions, clouds aim to provide elastic scalability.
\end{itemize}

Aim
\begin{itemize}
\item provide a platform on which vehicular functions may run ubiquitously, i.e., on vehicle hardware, and cloud infrastructures alike
\item facilitate computation offloading
\item ease data collection to enable big data innovation and to provide crowd-sourced services
\item Special requirements: Must deal with spotty reception
\item Therefore: Seamless switch between cloud and on-board
\item System needs to be modular to allow for fine-grained control over which functionality is offloaded
\end{itemize}

Method
\begin{itemize}
\item  Conceptual framework is presented which enables a service-oriented, vehicular software architecture that plays well with vehicular resource constrained on-board system.
\item the system is able to replicate services and run instances in the vehicle and in the cloud
\item SOA as means to modularize vehicular functions 
\item Containerization as means to make services portable
\item Middlewares based on the publish-subscribe communication paradigm excel in dealing with complexity. They feature location transparency. Thus, they are promising contenders. 
\item DDS as messaging middleware suitable for cyber-physical systems due to its reliability features. (Data centricity, because...?)
\item to bridge the gap between vehicle and cloud, an overlay network based on VLAN encapsulation is suggested
\item  An exemplary implementation of this framework is then presented using state-of-the-art technologies. It is comprised of Docker for containerization, and DDS as messaging middleware
\item the approach is then analyzed and discussed. Benchmarks are conducted.
\end{itemize}


Results
\begin{itemize}
\item special emphasis is put in reliability. It is feasible to create replicas of services running alongside in vehicle and cloud.
its reliability features are proven in one of the conducted benchmarks
\item lay ground for further innovations.
\end{itemize}
