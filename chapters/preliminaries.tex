

\chapter{Preliminaries}\label{chapter:preliminaries}

\section{Cyber-Physical Systems}

\begin{itemize}
\item characterized by limited resources
\item often real-time systems
\item often safety critical
\item Sensors and actors connected to ECUs
\item Examples of a cyber-physical system are vehicles
\item x-by-wire
\end{itemize}


\paragraph{Automotive System Engineering}
Control systems in modern vehicles are typically implemented as a collection of dozens, if not hundreds, of Electronic Control Units (ECUs) dispersed within a vehicle. Originally, there was no conscious decision to design automotive systems that way. Much rather, it is the result of an evolutionary process. At first, individual ECUs, each dedicated to a single purpose, were implemented in vehicles. At this point, ECUs were isolated from each other and no sense of cohesion was present in the system. This changed with the introduction of advanced wiring and bus systems that allowed ECUs to interact with sensors and actors. It was then only a matter of time until the bus systems were used to furthermore interconnect ECUs, which could then be used in interplay to create new, innovative functions \cite{broy2006challenges}. However random this evolution might have been, there are several benefits to the dispersed approach (as opposed to a centralized one). By having ECUs close to the sensors and actors they control, wiring effort is kept low, which results in low transmission latencies. 

Problems according to \citeauthor*{broy2006challenges}:
\begin{itemize}
\item Often highly proprietary
\item Limited re usability of software: 90\% of software is re-written 
\item Lack of tools and automation
\end{itemize}


Industry profile according to \citeauthor*{broy2006challenges}:
\begin{itemize}
\item Highly modular: several teams working independently on different technologies
\item Much is outsourced: many technologies are developed by suppliers, rather than the OEM 
\item Development: Many systems must interact -> vehicles evolve from an assembled device to an integrated system
\item Behavior becomes programmable: from comfort functions to steering and breaking: everything can be controlled by software.
\item moving away from specialized ECUs to general-purpose commodity systems
\end{itemize}

%
%
%
%
%
%
%
%
%
%

\section{Distributed Systems}
Modern vehicular E/E-Architectures\footnote{Electric/Electronic} are comprised of a large number of distributed, connected sensors, actors and control units, and hence, fall into the category of \emph{distributed systems}. The term "distributed system" entails many things and just as many definitions of the term exist. A definition that most would agree upon is the one given by \citeauthor*{tanenbaum2017distributed} \cite{tanenbaum2017distributed}: 
\begin{quote}
"A distributed system is a collection of autonomous computing elements that appears to its users as a single coherent system."
\end{quote}

The first aspect to consider in this definition is the word \emph{collection}. Distributed systems are made up of a number of \emph{nodes} which may occur in the form of either hardware devices, or software processes. Nodes work together to achieve a common goal. For this, they need to exchange messages (\cf \autoref{sec:middlewares}). Furthermore, the definition names \emph{autonomy} as a characteristic of distributed systems. Nodes, on their own, are autonomously acting entities, with their own, individual sets of rules and behavior. At the same time the system needs to be kept together. \emph{Groups}, which individual nodes may join, are a tool to achieve this. There are open groups, which every node may join, and closed groups, which employ an authorization mechanism to control access. 
Groups aim to provide \emph{coherence}, which is another aspect of the definition given above. By that definition, however, the coherence of the system is only \emph{perceived}. \Ie , to users, whether they are humans or programs, a distributed system presents itself as a single entity, even though it is in actuality comprised of a number of physically dispersed processes and resources. This principle is called \emph{distribution transparency}. \citeauthor*{tanenbaum2017distributed} \cite{tanenbaum2017distributed} separate this principle into several aspects. The first one to note is \textbf{location transparency}. At the root of location transparency is the desire to hide the physical location of resources. A common method to achieve this is by giving resources names. A user who wants to access a resource can thus refer to it by name, \eg\ a URL, while remaining oblivious of its actual location. Under the hood, communication is still based on location-dependent addresses, but such details can be hidden by a name resolution service. Naming furthermore facilitates another kind of distribution transparency: \textbf{Relocation transparency}. As the name suggests, relocation transparency aims to hide the fact that resources may move without the user taking notice. In the example of the aforementioned name resolution service, this can be achieved by reconfiguring the service to redirect users to a location different than the one previously known. To the user, still, the resource appears to be in the same location as it only knows its name. Related to relocation transparency is \textbf{migration transparency}, but in contrast to the former, migration transparency refers to the mobility of the \emph{user}. A migration transparent system allows a user to roam freely, while maintaining connectivity to the rest of the system. Examples of such systems are cellular networks.

Another aspect of distribution transparency is \textbf{replication transparency}. Distributed systems often provide means to replicate nodes or resources, \eg\ to improve scalability. Replication transparency states that all such replicas appear as one to the user. In addition to scalability, replication can be helpful to provide failure resilience. If a given node fails, and a replica is available, the user can be automatically redirected to the replica. This is also known as \textbf{failure transparency}.

The last kind of distribution transparency that \citeauthor*{tanenbaum2017distributed} list is \textbf{access transparency}. Access transparency refers to how data is presented to different users. Several users may have entirely different views of the same data\todo{explain better}. At the basis of this is a basic principle of software engineering: the separation of data and its representation.

%
%
%
%
%
%
%
%
%
%

\section{Middlewares} \label{sec:middlewares}
A prerequisite for the coherence property of distributed systems is the need for nodes to engage in collaboration. More precisely, distributed applications need a way to pass messages, or data, between different threads of execution. For this purpose, \emph{middlewares} \cite{bernstein1996middleware} are commonly used. Although message passing is a prime example of a middleware's use case, there are many other important concepts for which middlewares exist, \eg , transaction management in database systems. The primary goal of middlewares is to abstract away such concepts so that programmers can focus on implementing business logic and shipping features, instead of having to deal with the underlying specifics. Middlewares are implemented as a software layer that sits between operating system and the actual applications. They are often included in the form of libraries that make the middleware's functionality available to the programmers by means of an Application Programming Interface (API). 

The need for middlewares becomes particularly evident in the example of \emph{messaging}. Messages need to be sent over a physical mediums which are often, by nature, unreliable. To ensure reliability, many things need to be taken care of, \eg , error detection, repetition mechanisms, etc. In addition, guarantees must be given that messages are delivered to the right receivers. Therefore, addressing and routing mechanisms must be in place. These are just a few examples of hard-to-solve problems related to messaging. Managing these things manually, and implementing according measures from the ground up is hardly feasible for programmers. Messaging middlewares are a good way to streamline this process. An exemplary messaging middleware may be included in a software project in the form of a library which exposes an API. In the following, all the programmer needs to do to send messages is to invoke a simple function, \eg , \texttt{middleware.sendBroadcast("Hello, World!")}. The middleware then ensures that the message is delivered to the right recipients over whichever transport is available. To receive messages, the programmer may, in the case of this exemplary middleware, define a callback function which the middleware calls automatically whenever a new message is available. The callback function could have the form \texttt{void receive(string message, string sender) \{ /*...*/ \}}.

%
%
%
%
%
%
%
%
%
%

\section{Networking}

\subsection{Overlay Networks}
Distributed systems are often organized as \emph{overlay networks}\footnote{The term "overlay networks" is often used interchangeably with its abbreviated form, "overlays"} \cite{tarkoma2010overlay}. An overlay network is a logical network which connects nodes, or peers, in an abstract, high-level manner. Naturally, in order to enable information exchange between the peers, overlay networks require a substrate physical network (\emph{underlay}) over which data can be transmitted. As opposed to physical networks, which connect \emph{physical machines}, overlay networks connect \emph{processes}. An important thing to note is that overlay networks are entirely decoupled from the physical infrastructure, such that both networks may evolve (change topology) independently without affecting their operability \cite{tanenbaum2017distributed}. For example, an added node in an overlay network does not necessarily entail the addition of a physical node. Conversely, the removal of a physical node does not necessarily result in connection loss of a logical node. An example of this is the Internet \cite{vaezi2017virtualization}, which spans a worldwide network of nodes that is resilient to failures, such that, when a physical node breaks, a redundant path to the target node may be taken. Other examples of overlay networks are VPNs, Peer-to-peer (P2P) networks and voice over IP (VoIP).

There are two types of overlay networks: \emph{structured} and \emph{unstructured} overlay networks. In the former, peers are organized in a specific, deterministic manner, such that each node has its firm place and an immutable set of neighbors. Unstructured overlay networks, on the other hand, allow the topology to change dynamically. In order for this to work, each node maintains an ad-hoc list of neighbors that is to be updated continuously \cite{tanenbaum2017distributed}. In the context of this thesis, unstructured overlay networks are of particular interest due to the dynamic nature and the reliability characteristics of mobile systems. 

Different ways exist to create overlays. The one this thesis is concerned with is the \emph{Ethernet virtualization}-type of overlays, which are created by using IETF VXLAN\footnote{"Virtual eXtensible Local Area Network"}. 


Ethernet virtualization: VLAN tags to separate traffic and to build L2/L3 overlays.


\subsection{Multicast}
The simplest form of communication is one-to-one communication, which is also referred to as \emph{unicast}. In unicast, every node has an address by which it can be contacted. In most cases, unicast is sufficient for data exchange. However, in distributed systems, information frequently needs to be propagated to multiple receivers simultaneously. Unsurprisingly, this type of communication is called \emph{multicast} communication. 

benefits: No hard-coded addresses

Multicast can be implemented on both, network and application-level.

Support for multicast in WANs is rather limited.

%
%
%
%
%
%
%
%
%
%

\section{Cloud Computing}

The idea of cloud computing is to provide access to remote computing resources  (\eg , networks, servers, storage, applications, and services) in a convenient, on-demand manner \cite{mell2011nist}. Customers of cloud providers can rent these resources to use them at their will. By outsourcing their IT infrastructure into the cloud, companies can significantly reduce capital and operational expenditures (CAPEX/OPEX) that are typically associated with running an on-premise infrastructure.\todo{citation} Other benefits include easier maintenance and accelerated time-to-market times. Several pricing models for cloud services exist. Customers often have the choice between a set monthly fee, or they may take advantage of a pay-per-use model, whereby providers bill their users, \eg , on the basis of CPU time.

Cloud infrastructures are typically implemented as multi-tenancy systems in which the same hardware is shared among many customers ("resource pooling"). An enabling technology for this is virtualization. Each customer is assigned one or more virtual machines (VMs) running on one or more physical servers. For its users, the alloted computing environment appears as a single, isolated physical machine. The amount of disposable resources can be controlled for each VM individually, allowing for fine-grained resource tuning according to demand.

A major selling point of cloud computing is scalability. Many cloud providers allow for the dynamic allocation of resources depending on demand. To customers, the available resources appear as if they were unlimited when in actuality the substrate resources are alloted and released under the hood in an elastic manner. To steer this behavior, \emph{elasticity controllers} can be employed which allow customers to define rules to control when and how scaling measures are performed. \Eg , when a CPU utilization threshold is reached, the system can be instructed to automatically launch an application replica. Subsequently, a load balancer can be used to distribute the load between the instances \cite{vaquero2011dynamically}.


\paragraph{}
Cloud computing presents itself in the form of several usage models. The most notable ones are: \emph{Infrastructure as a Service} (IaaS),  \emph{Platform as a Service} (PaaS),  and \emph{Software as a Service} (SaaS) \cite{mell2011nist}.

\begin{description}
\item[IaaS] In the IaaS model, sheer, usually virtualized, hardware is provided, on which customers can install and run arbitrary software. Customers have only basic control over their virtual infrastructure's hardware composition, but may take influence on the operating system level.
\item[PaaS] The PaaS model presents a higher level view on the infrastructure. In this model, customers don't have full control over their VM instance. Instead, they can deploy their software in predefined application-hosting environments \cite{mell2011nist} which are typically centered around a certain technology, \eg\ .NET, or node.js, etc. Examples for this type of services are Microsoft Azure or Google App Engine.
\item[SaaS] Finally, the SaaS model is the highest level of cloud services.\todo{sounds weird} SaaS typically describes applications which are hosted on cloud platforms. These applications are usually accessible by means of thin clients, most notably web browsers. Customers have the least control over the cloud service and can only take influence via application-level configurations \cite{mell2011nist}. Examples of SaaS applications are browser-based e-mail services or video streaming platforms.
\end{description}


\section{Real-time Systems}

\section{Safety-critical Systems}
A system is considered emph{failed} when it cannot meet its promises. [Tanenbaum, S 426]

An \emph{error} may lead to system failure.

A \emph{fault} is the cause of an error. Faults can be caused by bugs in the software or unforeseeable circumstances. Some faults are avoidable and some are out of the control of the system developers. Nevertheless, all faults need to be dealt with.

A \emph{transient fault} is one that affects the system temporarily and can be circumvented by repeating the same operation.

An \emph{intermittent fault} occurs on and off, seemingly out of nowhere, without being reproducible .

A \emph{permanent fault} is one that persists until the root of the fault is found and repaired.

\emph{Fault tolerance} is a systems ability to continue operation, even in the presence of faults.

Failure masking by redundancy: three types: information r. (hamming code), time r. (repeat error), physical r. (redundant processes or hardware)



\section{Containerization with Docker}

In recent years, container technology has gained widespread adoption in the software development world. By greatly simplifying the software deployment—and development workflow, containers have become a cornerstone of successful software architectures.
Considering their advantages over hypervisor-based virtualization, such near native performance, sub-second boot times \cite{felter2015updated}\cite{morabito2015hypervisors} and minimal disk space usage, containers bring qualities to the table that are relevant specifically to embedded systems.
Containers are independent units of deployment containing everything a particular application needs to run from system libraries, tools, and runtimes to application specific settings. 
Akin to hypervisor-based virtualization, Linux containers aim to provide isolated, self-contained execution environments for applications that may be moved freely between hosts without affecting the application’s behavior.

Unlike traditional virtual machines, however, applications in containers run on non-virtualized hardware, and thereby, with minimal performance overhead when properly configured \cite{felter2015updated}\cite{morabito2015hypervisors}. This is of interest particularly in the field of embedded systems where resource constraints are prevalent.
Enabled by the concept of \emph{kernel namespaces}, multiple containers may run on the same host at the same time without affecting one another, or even having knowledge of each others existence. 
Namespaces wrap a set of system resources and present them to the container process as if they were dedicated to it. Each aspect of a container runs in its own namespace and its access is limited to that namespace. Hence, a level of isolation is achieved that was previously only possible with virtual machines. The only thing containers share is the host’s OS kernel, and optionally, parts of the file system.

To which extent a container may use the host system's resources is controlled through a mechanism called \emph{cgroups}, which is shorthand for \emph{control gorups}. 
Cgroups is a Linux kernel feature that allows to limit the resources availabe to a group of processes. Containers, effectively being groups of processes, may therefore be alloted a certain amount of computing resources. This allows for fine-grained control over the resource utilization of individual containers running on a host system.

Docker \cite{DockerWebsite} is undoubtably the most prominent container technology and arguably the first one to make Linux containers accessible for general use. 

\subsection{Docker Images}
Unique to Docker is its approach to container images. Images may be seen as the ``blueprint'' on the basis of which containers are built. 
Images are implemented utilizing UnionFS, a Linux service that facilitates the layering of different file systems atop each other. 
Leveraging this technology, Docker images are made of layers, with each layer adding to, or modifying, the respective underlying layer. 
A benefit of this approach is that individual layers may be shared and reused in other containers, thereby saving tremendous amounts of disk space compared to traditional VM images.

\subsection{Dockerhub}

\subsection{Docker Networking}


% https://thenewstack.io/container-networking-breakdown-explanation-analysis/

\emph{libnetwork}

Docker provides several methods to create network links between containers.  



\paragraph{Bridge.} By default, \docker\ connects containers via a Linux bridge. Bridges are host-internal network interfaces. Through \texttt{iptables} functions like NAT\footnote{Network Address Translation} and port mapping are facilitated.

\paragraph{Host.} A Container in \emph{host} mode takes over the network interface of the host system. Therefore, all capabilities that that the host system possesses also apply to that container. A disadvantage of this is that only one container may run on a given host at any time.

\paragraph{Overlay.} 
Overlay networks \cite{tarkoma2010overlay} are ...
VXLAN as tunneling technology.
Serf as gossip protocol.
Implementations: flannel.
Benefits: Cross cloud connectivity, no public ports.

\paragraph{Underlay.}
MACvlan: Separate MAC and IP address assigned to each container. Eliminates the need for bridges and NAT, making it performant. Containers are entirely isolated from the host, increasing security.
IPvlan: Similar to MACvlan but instead of having one MAC address per container, all containers on a host share the same address. This works around a common security measure in network switches to prohibit the use of multiple MAC addresses per physical port.



\section{Data Distribution Service}

DDS is a messaging middleware standard \cite{dds-1.4-standard} for distributed applications. The standard is designed for mission- and business critical systems with real-tme requirements. As such, it aims to function in a resource efficient and predictable manner, succumbing to minimal computational and transport overhead.

Specifies an API by which a distributed application can pass data over DCPS. 



DDS is built around the data-centric publish-subscribe paradigm.
Data centric means....

In the publish-subscribe paradigm, two kinds of peers are present: publishers and subscribers. Publishers offer data, while subscribers subscribe to receive that data. A crucial characteristic of publish-subscribe is that data exchange between the peers is anonymous. \Ie , subscribers do not know where a given message originated. The same is true for publishers: they have no knowledge about where the sent data will end up at -- or even if there are any receivers. Thus, there is no concept of \emph{direct addressing}. Instead, peers communicate on the basis of a shared understanding of what \emph{kind of data} they are interested in. 
For example, given a temperature sensor offering temperature measurements in a publish-subscribe setting. A subscriber that is interested in that data only knows that it wants to receive \emph{temperature data} while at the same time being entirely oblivious to the concept of \emph{temperature sensors}. 
Since publishers and subscribers have no references to one another, and know as little of each other as possible, a high level of loose coupling is achieved. This allows for a simple extension of the system, making it extraordinarily scalable.

By abstracting away the source of data, a \emph{virtual global data space} is created. Each component connected to the system views data as if it were available in a local storage, when in reality, it is distributed.


\paragraph{Programming Interface.}
The DDS specification is split into two separate sections. The main one, which is concerned with \emph{Data-Centric Publish-Subscribe} (DCPS), defines a low level API that enables applications to communicate via DDS. The second part revolves around a \emph{Data Local Reconstruction Layer} (DLRL). DLRL sits on top of DCPS and is optional. The purpose of DLRL is to provide typed interfaces to the messaging layer, \ie , the delivered messages are conceived in a format suitable for direct processing in the application--without the need to check the message's format. More precisely, DCPS performs a transformation of the unprocessed messages into language-specific data types. With the aid of DLRL, type-safety of communication is ensured and verification can be performed at compile-time, thereby reducing the application's error-proneness.


\paragraph{Wire Protocol.}
At the base of DDS, there is a wire protocol deliberately tailored to DCPS-style communication: Real-time Publish-Subscribe (RTPS). Although RTPS is related to DDS, its specifics are outsourced in a separate standard. \cite{rtps-2.2-standard}

Goals: provide platform independent means of communication to make DDS implementations interoperable.



\paragraph{Bla}
Offers transport transparency.

data-centric instead of message-centric. The difference is that the former implies a shared data model. The middleware has an understanding of the data and its context and is responsible that all components have a common view of the data.
The advantage of data-centric messaging is that it allows a higher abstraction. Developers can focus on the data itself and on developing business logic instead of having to implement data sharing through exchange of messages.

DDS is a message bus. This is in contrast to a broker-based architecture. A broker enables flexible routing patterns featuring filtering, variable numbers of message queues etc. However, it can be considered a single point of failure.

Being a standard, DDS strives for interoperability between implementations


\begin{figure}[htpb]
  \centering
  \includegraphics[width=\textwidth]{figures/dds.pdf}
  \caption[DDS example]{An example of a distributed application connected by means of DDS}\label{fig:dds}
\end{figure}

\subsection{DDS Components}

DDS defines a number of components, for which it uses its own nomenclature. In the following, each component is described. \autoref{fig:dds} shows how they are related.


\paragraph{Topics.}


\paragraph{Publishers and Data Writers.}


\paragraph{Subscribers and Data Readers.}

\paragraph{Domains and Domain Participants.}
At the highest level, there are \emph{Domains}. Domains are the DDS way of grouping together sets of coherent \emph{domain participants} and to separate those sets from each other. Speaking in terms of distributed systems, domains are a mechanism to manage group memberships of nodes. \cite{tanenbaum2017distributed}

Domain participants are entities that belong to a particular domain. Each publisher, subscriber and topic is derived from one domain participant and is therefore dedicated to exactly that domain. As a consequence, participants of different domains are entirely separated from each other.   

Depicted in \autoref{fig:dds} is only one domain. However, there could just as well be other domains. 



\subsection{Concepts}

\paragraph{Data Centricity.}


\paragraph{Dynamic Service Discovery.}
Offers dynamic service discovery.


\paragraph{Quality of Service.}
One of DDS's salient features is its intrinsic QoS support implemented by \emph{QoS policies}.
QoS policies specify service attributes for controlling each participant's behavior and quality properties. They can be set for each participant and topic individually. An example for a QoS policy is the \texttt{DEADLINE} policy. It specifies the minimum message frequency of a service. If the deadline period of a hypothetical data writer is set to, e.g., 100 ms, that means that this data writer is required to send a message at least every 100 ms. If it fails to send a message at this rate, the data writer and all the respective topic's readers will need to deal with this circumstance on a code level. 

In addition, QoS policies serve as service contracts. They specify non-functional requirements that services must fulfill to be able to communicate with each other. E.g., a service provider's \texttt{RELIABILITY} policy may have been set to the \texttt{BEST\_EFFORT} level, thereby allowing the service to drop samples. A service consumer, on the other hand, may require the service provider's policy to be set to \texttt{RELIABLE}, which prohibits the dropping of samples. Since the service provider only insufficiently fulfills the service consumer's QoS requirements, the services are considered incompatible with each other. QoS policies can therefore be seen as service contracts on a technical level, specifying service compatibility. It may be noteworthy, however, that these contracts do not rid the need for proper interface contracts modeled by a service designer.

Despite their name, QoS policies do not only concern quality attributes. They can also be used to specify the priority of messages, their lifespan, i.e. how long they are valid, or how many messages are kept in local memory.



\paragraph{Data Centricity.}

\paragraph{Asynchronous Messaging.}


\paragraph{Location Transparency.}

\paragraph{Decentralization.}

\paragraph{Platform Independence.}


\paragraph{Limitations.}
Requires full fledged operating system which can not be guaranteed in embedded systems.

OpenDDS: Up to 120 domain participants


\subsection{DDS for Automotive Systems}
Automotive software systems have previously relied -- and, to some degree, will continue to do so -- on low-level, low-bandwidth transport protocols such as CAN, LIN, FlexRay, etc. Up until now, networks stacks based on those protocols were sufficient to meet the basic requirements of delivering vehicular sensor data. However, in the future, more and more data will be collected within vehicles and more and more data will be required to feed into intelligent systems such as ADAS. These systems increasingly rely on high volumes of data from video cameras, or LIDARs. As a result, bandwidth requirements for vehicle-intrinsic computer networks are skyrocketing. At the same time, these functions require computational capabilities that go way beyond of what is possible with the microcontrollers typically used in traditional ECUs. High-performance computer systems based on high-level operating systems are needed to meet the new requirements.


Hence, there is a need for 

DDS is designed for resource constrained real-time applications such as sensor networks or industrial automation.

DDS allows to configure how much of a system's resources an DDS-enabled application may use. Consequently, it is the middleware's responsibility to allocate resources as needed while still staying within the specified boundaries. At the same time, priorities aligning with the application's QoS settings need to be considered. DDS takes this burden off the programmer's shoulders.

Predictable


\subsection{Implementations}
As mentioned, DDS, in itself, is only a standard. As such, DDS does not dictate, in detail, how to implement the concepts presented in the earlier sections. A number of DDS implementations by different vendors exist, all varying in terms of standard fulfillment, features beyond the standard, and pricing. Compatibility between the respective implementations is ensured through the \emph{DDS Interoperability Protocol} (DDSI). It uses the OMG \emph{Common Data Representation} (CDR) to encode data in a platform-neutral way. 

\paragraph{OpenDDS.}
Two types of discovery: centralized Information Repository, distributed RTPS discovery. The latter must be used if DDS implementation compatibility is priority

Only supports C++ and Java

\paragraph{OpenSplice.}


\paragraph{RTI.}
out of the implementations available to the broad public, it is by far the most mature and feature rich implementation.
Features encryption, compliance to several safety standards

\paragraph{Others.}
Miltech