

\chapter{Preliminaries}\label{chapter:preliminaries}


\section{Distributed Systems}

"A distributed system is a collection of autonomous computing elements that appears to its users as a single coherent system" \cite{tanenbaum2017distributed}

made up of a number of hardware devices and/or software processes called \emph{nodes}

there are open groups, which every node may join, and closed groups, which employ an authorization mechanism to control access.

\emph{Distribution transparency}: See Tanenbaum section 1.2

A requirement for distributed applications is the need to pass messages, or data, between different threads of execution. For this purpose, \emph{middlewares} \cite{bernstein1996middleware} are commonly used. 


How are cars distributed systems?

\paragraph{Challenges} In a distributed system, single parts may fail while others keep functioning properly.  This may lead to unexpected behavior.


\section{Real-time Systems}



\section{Containerization with Docker}

In recent years, container technology has gained widespread adoption in the software development world. By greatly simplifying the software deployment—and development workflow, containers have become a cornerstone of successful software architectures.
Considering their advantages over hypervisor-based virtualization, such near native performance, sub-second boot times \cite{felter2015updated}\cite{morabito2015hypervisors} and minimal disk space usage, containers bring qualities to the table that are relevant specifically to embedded systems.
Containers are independent units of deployment containing everything a particular application needs to run from system libraries, tools, and runtimes to application specific settings. 
Akin to hypervisor-based virtualization, Linux containers aim to provide isolated, self-contained execution environments for applications that may be moved freely between hosts without affecting the application’s behavior.

Unlike traditional virtual machines, however, applications in containers run on non-virtualized hardware, and thereby, with minimal performance overhead when properly configured \cite{felter2015updated}\cite{morabito2015hypervisors}. This is of interest particularly in the field of embedded systems where resource constraints are prevalent.
Enabled by the concept of \emph{kernel namespaces}, multiple containers may run on the same host at the same time without affecting one another, or even having knowledge of each others existence. 
Namespaces wrap a set of system resources and present them to the container process as if they were dedicated to it. Each aspect of a container runs in its own namespace and its access is limited to that namespace. Hence, a level of isolation is achieved that was previously only possible with virtual machines. The only thing containers share is the host’s OS kernel, and optionally, parts of the file system.

To which extent a container may use the host system's resources is controlled through a mechanism called \emph{cgroups}, which is shorthand for \emph{control gorups}. 
Cgroups is a Linux kernel feature that allows to limit the resources availabe to a group of processes. Containers, effectively being groups of processes, may therefore be alloted a certain amount of computing resources. This allows for fine-grained control over the resource utilization of individual containers running on a host system.

Docker \cite{DockerWebsite} is undoubtably the most prominent container technology and arguably the first one to make Linux containers accessible for general use. 

\subsection{Docker Images}
Unique to Docker is its approach to container images. Images may be seen as the ``blueprint'' on the basis of which containers are built. 
Images are implemented utilizing UnionFS, a Linux service that facilitates the layering of different file systems atop each other. 
Leveraging this technology, Docker images are made of layers, with each layer adding to, or modifying, the respective underlying layer. 
A benefit of this approach is that individual layers may be shared and reused in other containers, thereby saving tremendous amounts of disk space compared to traditional VM images.

\subsection{Dockerhub}

\subsection{Docker Networking}


% https://thenewstack.io/container-networking-breakdown-explanation-analysis/

\emph{libnetwork}

Docker provides several methods to create network links between containers.  



\paragraph{Bridge.} By default, \docker\ connects containers via a Linux bridge. Bridges are host-internal network interfaces. Through \texttt{iptables} functions like NAT\footnote{Network Address Translation} and port mapping are facilitated.

\paragraph{Host.} A Container in \emph{host} mode takes over the network interface of the host system. Therefore, all capabilities that that the host system possesses also apply to that container. A disadvantage of this is that only one container may run on a given host at any time.

\paragraph{Overlay.} 
Overlay networks \cite{tarkoma2010overlay} are ...
VXLAN as tunneling technology.
Serf as gossip protocol.
Implementations: flannel.
Benefits: Cross cloud connectivity, no public ports.

\paragraph{Underlay.}
MACvlan: Separate MAC and IP address assigned to each container. Eliminates the need for bridges and NAT, making it performant. Containers are entirely isolated from the host, increasing security.
IPvlan: Similar to MACvlan but instead of having one MAC address per container, all containers on a host share the same address. This works around a common security measure in network switches to prohibit the use of multiple MAC addresses per physical port.


\section{Networking}

\subsection{Multicast}

No hard-coded addresses



\subsection{Container Networking with \wnet}

\wnet\ is an open source project\footnote{\url{www.github.com/weaveworks/weave}} which aims to provide advanced overlay networking capabilities for \docker\ containers. It is designed to make up for the shortcomings of \docker 's built-in overlay networks, and more specifically, their lack of encryption and multicast support. By means of \docker\ overlay networks, containers dispersed among physical hosts that reside in their own physical networks, may exchange data freely. \wnet\ presents the network to the applications in a location transparent way, such that from the applications' viewpoint, it does not matter whether its peers are located on the same host or within a data center on the other side of the world. \wnet\ is developed by a global team, mostly employed by the London-based software company \emph{Weaveworks}\footnote{\url{www.weave.works}}.


\paragraph{Functioning.}
\wnet\ is implemented as client software which needs to be installed on each machine that is supposed partake in the overlay. The software can be started using a single command, and in the following, all containers launched on that host will automatically connected to the overlay network. This is achieved by means of a \emph{\docker\ API proxy}. The proxy sits between \docker 's command-line client and the \docker\ daemon and intercepts all communication between the two. When the \docker\ engine is instructed to start a container, the proxy takes all precautions needed to enable overlay networking for that container. Once the connection is established, all container traffic is routed through three dedicated network channels: one TCP connection to exchange meta data about the network, and two UDP channels for duplex data exchange.


\begin{figure}[htpb]
  \centering
  \includegraphics[width=\textwidth]{figures/sdn.pdf}
  \caption[An example of containers connected via \wnet\ overlay network]{A number of dispersed containers connected by a \wnet\ overlay network}\label{fig:weavescheme} \todo[inline]{Add legend for boxes, add another host}
\end{figure}

When the Weave software is started, a central component of \wnet , the \emph{Weave router} is launched (\cf\ \autoref{fig:weavescheme}). Similar to a hardware router, the Weave router is responsible for the forwarding and routing of data packets to their appropriate receivers. Weave routers can be seen as gateways through which all containers participating in a Weave network are connected. To facilitate routing on the data plane, a custom UDP encapsulation protocol, called \emph{sleeve}, was devised. A Weave router in itself is an containerized application running at all times, in the same way a daemon would. There is one of such router containers running on every host in a Weave-enabled infrastructure.

The Weave router is a user space process. As such, a context switch is needed every time it is tasked to process a packet. This comes with a substantial performance overhead. Hence, as a faster alternative, the so-called \emph{fast datapath} mode was added. In this mode, packets are processed by the Linux kernel instead of by the Weave router. This way, the context switch into user space is omitted. \wnet\ leverages the Linux kernel's \emph{Open vSwitch datapath} module \cite{pfaff2015design} to achieve this behavior. Open vSwitch can be used to create a software-based, virtual network switch. That way, the kernel can be instructed to process packets in a certain way. For instance, the kernel can be commanded to add a VXLAN header to each packet, thereby achieving the same result as the Weave router, but faster. However, fast datapath mode can only be used when the underlying infrastructure allows it. The Internet is a particular example of a network where fast datapath communication is hard to achieve.


\paragraph{Topology Management.} 
Weave's topology management is self-governing and self-healing. Peers continually exchange topology information and monitor the state of the network. Whenever peers lose connectivity, they continuously try to re-establish the connection until it is restored. All participating peers know the topology of the entire network. For this, \wnet\ employs a sophisticated discovery and topology management mechanism by which changes in the network topology are rapidly propagated within the network. The topology management protocol is based on a spanning-tree broadcast mechanism known from hardware switches. To further ensure that all peers have an up-to-date neighbor list at all times, \wnet\ additionally employs a custom neighbor gossiping protocol by which each peer sends update messages to a random subset of their neighbors. Updates of the network's topology are performed periodically as well as when certain events occur, such as when a node joins or leaves the network.


\paragraph{Encryption.} 
As mentioned earlier, a salient feature of \wnet\ is the ability to encrypt all traffic within the overlay network. This is especially important for networks that span over insecure infrastructures, such as the Internet. To set up encryption, a shared secret (password) needs to be provided when the Weave router is launched. The shared secret must therefore be present on all participating hosts. From the password, salted, ephemeral session keys are generated which are then used to encrypt the packets. For each connection between any two peers, one unique session key exists.

The way encryption is applied differs depending on the forwarding mode used (sleeve or fast datapath). In fast datapath mode, a IPsec-based protocol is used, whereby each packet is wrapped in an encapsulated security payload (ESP). Because each packet in this mode is processed by the Linux kernel, encryption is applied by means of the standard Linux Kernel Crypto API which is thoroughly tested, and generally considered secure.

For sleeve mode, a custom encryption algorithm based on TLS\footnote{Transport Layer Security} is used. As with fast datapath encryption, sleeve mode encryption utilizes shared, ephemeral session keys for each connection. \autoref{fig:weaveencryption} depicts how these session keys are generated. In the image, two hosts ($H_1$, $H_2$) are to establish an encrypted connection. First, $H_1$ initiates the key exchange by sending a handshake message to $H_2$ \circled{1}. Then, both hosts generate their own, individual key pairs so that each host has a public key and a private key. The key pair for $H_1$ is $(K_{1P}, K_{1S})$ and the key pair for $H_2$ is $(K_{2P}, K_{2S})$. Once that is done, both hosts exchange their respective public keys, $K_{1P}$, and $K_{2P}$ \circled{3}. Using the peer's public key and their own private key, both hosts derive an auxiliary shared key, $S_A$, by means of Diffie--Hellman key exchange \cite{bresson2001provably}: $D(K_{1S},K_{2P})$ \circled{4}. Finally, the actual shared key can be generated. For this, the shared password is appended to $S_A$ to provide authenticity. In order to bring the key to the desired length of 256 bit, the compounded key is additionally hashed via SHA256: $H(S_A, Q)$ \circled{5}. The end result of this procedure is the final shared key, $S_{12}$, which is then used to encrypt the traffic between the two hosts.

\begin{figure}[htpb]
  \centering
  \includegraphics[width=0.8\textwidth]{figures/weave-encryption.pdf}
  \caption[Weave's key exchange protocol]{\wnet 's key exchange protocol in sleeve mode}\todo[inline]{remove Q arrows, make circles white, S\_12 -> S}\label{fig:weaveencryption}
\end{figure}


\todo{move or delete}Drawbacks: \wnet\ does not support IPv6 which is to the detriment of IoT networks, which benefit from the vast number of addresses of IPv6. However, in the vehicle use case, it is rather unlikely that every ECU has its own IP address. Rather, there is one gateway connected to a 5G module and all traffic is routed through that gateway. Because of this, the address scarcity issue only peripherally applies to the intended use case.


\section{Continuous Integration and Delivery}



\section{Communication}

What has been discussed so far is what services are and how they can be implemented. One question that remains is how they are connected. 

Which communication patterns are there?

push vs. pull,
unicast vs. multicast
synchronous vs. asynchronous
centralized vs. decentralized

What makes PubSub especially attractive?
PubSub enables loose coupling. No apriori knowledge of communication partners required. Hardware independence: Services may be migrated between computing nodes. Abstraction of underlying infrastructure is important.
No startup dependencies: The order in which services finish initializing is irrelevant. Especially important with ephemeral services.


\section{Data Distribution Service}

DDS is a messaging middleware standard \cite{dds-1.4-standard} for distributed applications. The standard is designed for mission- and business critical systems with real-tme requirements. As such, it aims to function in a resource efficient and predictable manner, succumbing to minimal computational and transport overhead.

Specifies an API by which a distributed application can pass data over DCPS. 



DDS is built around the data-centric publish-subscribe paradigm.
Data centric means....

In the publish-subscribe paradigm, two kinds of peers are present: publishers and subscribers. Publishers offer data, while subscribers subscribe to receive that data. A crucial characteristic of publish-subscribe is that data exchange between the peers is anonymous. \Ie , subscribers do not know where a given message originated. The same is true for publishers: they have no knowledge about where the sent data will end up at -- or even if there are any receivers. Thus, there is no concept of \emph{direct addressing}. Instead, peers communicate on the basis of a shared understanding of what \emph{kind of data} they are interested in. 
For example, given a temperature sensor offering temperature measurements in a publish-subscribe setting. A subscriber that is interested in that data only knows that it wants to receive \emph{temperature data} while at the same time being entirely oblivious to the concept of \emph{temperature sensors}. 
Since publishers and subscribers have no references to one another, and know as little of each other as possible, a high level of loose coupling is achieved. This allows for a simple extension of the system, making it extraordinarily scalable.

By abstracting away the source of data, a \emph{virtual global data space} is created. Each component connected to the system views data as if it were available in a local storage, when in reality, it is distributed.


\paragraph{Programming Interface.}
The DDS specification is split into two separate sections. The main one, which is concerned with \emph{Data-Centric Publish-Subscribe} (DCPS), defines a low level API that enables applications to communicate via DDS. The second part revolves around a \emph{Data Local Reconstruction Layer} (DLRL). DLRL sits on top of DCPS and is optional. The purpose of DLRL is to provide typed interfaces to the messaging layer, \ie , the delivered messages are conceived in a format suitable for direct processing in the application--without the need to check the message's format. More precisely, DCPS performs a transformation of the unprocessed messages into language-specific data types. With the aid of DLRL, type-safety of communication is ensured and verification can be performed at compile-time, thereby reducing the application's error-proneness.


\paragraph{Wire Protocol.}
At the base of DDS, there is a wire protocol deliberately tailored to DCPS-style communication: Real-time Publish-Subscribe (RTPS). Although RTPS is related to DDS, its specifics are outsourced in a separate standard. \cite{rtps-2.2-standard}

Goals: provide platform independent means of communication to make DDS implementations interoperable.



\paragraph{Bla}
Offers transport transparency.

data-centric instead of message-centric. The difference is that the former implies a shared data model. The middleware has an understanding of the data and its context and is responsible that all components have a common view of the data.
The advantage of data-centric messaging is that it allows a higher abstraction. Developers can focus on the data itself and on developing business logic instead of having to implement data sharing through exchange of messages.

DDS is a message bus. This is in contrast to a broker-based architecture. A broker enables flexible routing patterns featuring filtering, variable numbers of message queues etc. However, it can be considered a single point of failure.

Being a standard, DDS strives for interoperability between implementations


\begin{figure}[htpb]
  \centering
  \includegraphics[width=\textwidth]{figures/dds.pdf}
  \caption[DDS example]{An example of a distributed application connected by means of DDS}\label{fig:dds}
\end{figure}

\subsection{DDS Components}

DDS defines a number of components, for which it uses its own nomenclature. In the following, each component is described. \autoref{fig:dds} shows how they are related.


\paragraph{Topics.}


\paragraph{Publishers and Data Writers.}


\paragraph{Subscribers and Data Readers.}

\paragraph{Domains and Domain Participants.}
At the highest level, there are \emph{Domains}. Domains are the DDS way of grouping together sets of coherent \emph{domain participants} and to separate those sets from each other. Speaking in terms of distributed systems, domains are a mechanism to manage group memberships of nodes. \cite{tanenbaum2017distributed}

Domain participants are entities that belong to a particular domain. Each publisher, subscriber and topic is derived from one domain participant and is therefore dedicated to exactly that domain. As a consequence, participants of different domains are entirely separated from each other.   

Depicted in \autoref{fig:dds} is only one domain. However, there could just as well be other domains. 



\subsection{Concepts}

\paragraph{Data Centricity.}


\paragraph{Dynamic Service Discovery.}
Offers dynamic service discovery.


\paragraph{Quality of Service.}
One of DDS's salient features is its intrinsic QoS support implemented by \emph{QoS policies}.
QoS policies specify service attributes for controlling each participant's behavior and quality properties. They can be set for each participant and topic individually. An example for a QoS policy is the \texttt{DEADLINE} policy. It specifies the minimum message frequency of a service. If the deadline period of a hypothetical data writer is set to, e.g., 100 ms, that means that this data writer is required to send a message at least every 100 ms. If it fails to send a message at this rate, the data writer and all the respective topic's readers will need to deal with this circumstance on a code level. 

In addition, QoS policies serve as service contracts. They specify non-functional requirements that services must fulfill to be able to communicate with each other. E.g., a service provider's \texttt{RELIABILITY} policy may have been set to the \texttt{BEST\_EFFORT} level, thereby allowing the service to drop samples. A service consumer, on the other hand, may require the service provider's policy to be set to \texttt{RELIABLE}, which prohibits the dropping of samples. Since the service provider only insufficiently fulfills the service consumer's QoS requirements, the services are considered incompatible with each other. QoS policies can therefore be seen as service contracts on a technical level, specifying service compatibility. It may be noteworthy, however, that these contracts do not rid the need for proper interface contracts modeled by a service designer.

Despite their name, QoS policies do not only concern quality attributes. They can also be used to specify the priority of messages, their lifespan, i.e. how long they are valid, or how many messages are kept in local memory.



\paragraph{Data Centricity.}

\paragraph{Asynchronous Messaging.}


\paragraph{Location Transparency.}

\paragraph{Decentralization.}

\paragraph{Platform Independence.}


\paragraph{Limitations.}
Requires full fledged operating system which can not be guaranteed in embedded systems.

OpenDDS: Up to 120 domain participants


\subsection{DDS for Automotive Systems}
Automotive software systems have previously relied -- and, to some degree, will continue to do so -- on low-level, low-bandwidth transport protocols such as CAN, LIN, FlexRay, etc. Up until now, networks stacks based on those protocols were sufficient to meet the basic requirements of delivering vehicular sensor data. However, in the future, more and more data will be collected within vehicles and more and more data will be required to feed into intelligent systems such as ADAS. These systems increasingly rely on high volumes of data from video cameras, or LIDARs. As a result, bandwidth requirements for vehicle-intrinsic computer networks are skyrocketing. At the same time, these functions require computational capabilities that go way beyond of what is possible with the microcontrollers typically used in traditional ECUs. High-performance computer systems based on high-level operating systems are needed to meet the new requirements.


Hence, there is a need for 

DDS is designed for resource constrained real-time applications such as sensor networks or industrial automation.

DDS allows to configure how much of a system's resources an DDS-enabled application may use. Consequently, it is the middleware's responsibility to allocate resources as needed while still staying within the specified boundaries. At the same time, priorities aligning with the application's QoS settings need to be considered. DDS takes this burden off the programmer's shoulders.

Predictable


\subsection{Implementations}
As mentioned, DDS, in itself, is only a standard. As such, DDS does not dictate, in detail, how to implement the concepts presented in the earlier sections. A number of DDS implementations by different vendors exist, all varying in terms of standard fulfillment, features beyond the standard, and pricing. Compatibility between the respective implementations is ensured through the \emph{DDS Interoperability Protocol} (DDSI). It uses the OMG \emph{Common Data Representation} (CDR) to encode data in a platform-neutral way. 

\paragraph{OpenDDS.}
Two types of discovery: centralized Information Repository, distributed RTPS discovery. The latter must be used if DDS implementation compatibility is priority

Only supports C++ and Java

\paragraph{OpenSplice.}


\paragraph{RTI.}
out of the implementations available to the broad public, it is by far the most mature and feature rich implementation.
Features encryption, compliance to several safety standards

\paragraph{Others.}
Miltech