\chapter{Realization} \label{chapter:realization}
After having discussed the basic concept of the approach in the previous chapter, in this chapter, an exemplary realization is presented. Previously, three key challenges were identified:
\begin{inparaenum}[(i)]
  \item \emph{reliable, anonymous information exchange},
  \item \emph{isolation}, and
  \item \emph{connectivity} (\cf \ref{sec:challenges}).
\end{inparaenum}
To tackle these challenges, three technologies are proposed:

\begin{enumerate}[(i)]
\item \textbf{DDS} to enable \emph{reliable, anonymous information exchange},
\item \textbf{\docker} as containerization tool to achieve \emph{isolation} and portability of services, and
\item \textbf{\wnet} as means to provide \emph{connectivity} between the containerized services.
\end{enumerate}

In the following, these three tools are presented and it is described how they are leveraged to realize the proof of concept implementation.


\section{Data Distribution Service}

DDS is a messaging middleware standard \cite{dds-1.4-standard} for distributed applications. The standard is designed for mission- and business critical systems with real-tme requirements. As such, it aims to function in a resource efficient and predictable manner, succumbing to minimal computational and transport overhead.

Specifies an API by which a distributed application can pass data over DCPS. 



DDS is built around the data-centric publish-subscribe paradigm.
Data centric means....

In the publish-subscribe paradigm, two kinds of peers are present: publishers and subscribers. Publishers offer data, while subscribers subscribe to receive that data. A crucial characteristic of publish-subscribe is that data exchange between the peers is anonymous. \Ie , subscribers do not know where a given message originated. The same is true for publishers: they have no knowledge about where the sent data will end up at -- or even if there are any receivers. Thus, there is no concept of \emph{direct addressing}. Instead, peers communicate on the basis of a shared understanding of what \emph{kind of data} they are interested in. 
For example, given a temperature sensor offering temperature measurements in a publish-subscribe setting. A subscriber that is interested in that data only knows that it wants to receive \emph{temperature data} while at the same time being entirely oblivious to the concept of \emph{temperature sensors}. 
Since publishers and subscribers have no references to one another, and know as little of each other as possible, a high level of loose coupling is achieved. This allows for a simple extension of the system, making it extraordinarily scalable.

By abstracting away the source of data, a \emph{virtual global data space} is created. Each component connected to the system views data as if it were available in a local storage, when in reality, it is distributed.


\paragraph{Programming Interface.}
The DDS specification is split into two separate sections. The main one, which is concerned with \emph{Data-Centric Publish-Subscribe} (DCPS), defines a low level API that enables applications to communicate via DDS. The second part revolves around a \emph{Data Local Reconstruction Layer} (DLRL). DLRL sits on top of DCPS and is optional. The purpose of DLRL is to provide typed interfaces to the messaging layer, \ie , the delivered messages are conceived in a format suitable for direct processing in the application--without the need to check the message's format. More precisely, DCPS performs a transformation of the unprocessed messages into language-specific data types. With the aid of DLRL, type-safety of communication is ensured and verification can be performed at compile-time, thereby reducing the application's error-proneness.


\paragraph{Wire Protocol.}
At the base of DDS, there is a wire protocol deliberately tailored to DCPS-style communication: Real-time Publish-Subscribe (RTPS). Although RTPS is related to DDS, its specifics are outsourced in a separate standard. \cite{rtps-2.2-standard}

Goals: provide platform independent means of communication to make DDS implementations interoperable.



\paragraph{Bla}
Offers transport transparency.

data-centric instead of message-centric. The difference is that the former implies a shared data model. The middleware has an understanding of the data and its context and is responsible that all components have a common view of the data.
The advantage of data-centric messaging is that it allows a higher abstraction. Developers can focus on the data itself and on developing business logic instead of having to implement data sharing through exchange of messages.

DDS is a message bus. This is in contrast to a broker-based architecture. A broker enables flexible routing patterns featuring filtering, variable numbers of message queues etc. However, it can be considered a single point of failure.

Being a standard, DDS strives for interoperability between implementations


\begin{figure}[htpb]
  \centering
  \includegraphics[width=\textwidth]{figures/dds.pdf}
  \caption[DDS example]{An example of a distributed application connected by means of DDS}\label{fig:dds}
\end{figure}

\subsection{DDS Components}

DDS defines a number of components, for which it uses its own nomenclature. In the following, each component is described. \autoref{fig:dds} shows how they are related.


\paragraph{Topics.}


\paragraph{Publishers and Data Writers.}


\paragraph{Subscribers and Data Readers.}

\paragraph{Domains and Domain Participants.}
At the highest level, there are \emph{Domains}. Domains are the DDS way of grouping together sets of coherent \emph{domain participants} and to separate those sets from each other. Speaking in terms of distributed systems, domains are a mechanism to manage group memberships of nodes. \cite{tanenbaum2017distributed}

Domain participants are entities that belong to a particular domain. Each publisher, subscriber and topic is derived from one domain participant and is therefore dedicated to exactly that domain. As a consequence, participants of different domains are entirely separated from each other.   

Depicted in \autoref{fig:dds} is only one domain. However, there could just as well be other domains. 



\subsection{Concepts}

\paragraph{Data Centricity.}


\paragraph{Dynamic Service Discovery.}
Offers dynamic service discovery.


\paragraph{Quality of Service.}
One of DDS's salient features is its intrinsic QoS support implemented by \emph{QoS policies}.
QoS policies specify service attributes for controlling each participant's behavior and quality properties. They can be set for each participant and topic individually. An example for a QoS policy is the \texttt{DEADLINE} policy. It specifies the minimum message frequency of a service. If the deadline period of a hypothetical data writer is set to, e.g., 100 ms, that means that this data writer is required to send a message at least every 100 ms. If it fails to send a message at this rate, the data writer and all the respective topic's readers will need to deal with this circumstance on a code level. 

In addition, QoS policies serve as service contracts. They specify non-functional requirements that services must fulfill to be able to communicate with each other. E.g., a service provider's \texttt{RELIABILITY} policy may have been set to the \texttt{BEST\_EFFORT} level, thereby allowing the service to drop samples. A service consumer, on the other hand, may require the service provider's policy to be set to \texttt{RELIABLE}, which prohibits the dropping of samples. Since the service provider only insufficiently fulfills the service consumer's QoS requirements, the services are considered incompatible with each other. QoS policies can therefore be seen as service contracts on a technical level, specifying service compatibility. It may be noteworthy, however, that these contracts do not rid the need for proper interface contracts modeled by a service designer.

Despite their name, QoS policies do not only concern quality attributes. They can also be used to specify the priority of messages, their lifespan, i.e. how long they are valid, or how many messages are kept in local memory.



\paragraph{Data Centricity.}

\paragraph{Asynchronous Messaging.}


\paragraph{Location Transparency.}

\paragraph{Decentralization.}

\paragraph{Platform Independence.}


\paragraph{Limitations.}
Requires full fledged operating system which can not be guaranteed in embedded systems.

OpenDDS: Up to 120 domain participants


\subsection{DDS for Automotive Systems}
Automotive software systems have previously relied -- and, to some degree, will continue to do so -- on low-level, low-bandwidth transport protocols such as CAN, LIN, FlexRay, etc. Up until now, networks stacks based on those protocols were sufficient to meet the basic requirements of delivering vehicular sensor data. However, in the future, more and more data will be collected within vehicles and more and more data will be required to feed into intelligent systems such as ADAS. These systems increasingly rely on high volumes of data from video cameras, or LIDARs. As a result, bandwidth requirements for vehicle-intrinsic computer networks are skyrocketing. At the same time, these functions require computational capabilities that go way beyond of what is possible with the microcontrollers typically used in traditional ECUs. High-performance computer systems based on high-level operating systems are needed to meet the new requirements.


Hence, there is a need for 

DDS is designed for resource constrained real-time applications such as sensor networks or industrial automation.

DDS allows to configure how much of a system's resources an DDS-enabled application may use. Consequently, it is the middleware's responsibility to allocate resources as needed while still staying within the specified boundaries. At the same time, priorities aligning with the application's QoS settings need to be considered. DDS takes this burden off the programmer's shoulders.

Predictable


\subsection{Implementations}
As mentioned, DDS, in itself, is only a standard. As such, DDS does not dictate, in detail, how to implement the concepts presented in the earlier sections. A number of DDS implementations by different vendors exist, all varying in terms of standard fulfillment, features beyond the standard, and pricing. Compatibility between the respective implementations is ensured through the \emph{DDS Interoperability Protocol} (DDSI). It uses the OMG \emph{Common Data Representation} (CDR) to encode data in a platform-neutral way. 

\paragraph{OpenDDS.}
Two types of discovery: centralized Information Repository, distributed RTPS discovery. The latter must be used if DDS implementation compatibility is priority

Only supports C++ and Java

\paragraph{OpenSplice.}


\paragraph{RTI.}
out of the implementations available to the broad public, it is by far the most mature and feature rich implementation.
Features encryption, compliance to several safety standards

\paragraph{Others.}
Miltech

%
%
%
%
%
%
%
%
%
%
\pagebreak
\section{\docker\ for Containerization}
In this thesis' approach, services and their dependencies shall be encapsulated in their own, self-contained execution environments that may be deployed inside the vehicle and also in the cloud. Containerization is a promising technology to achieve this. For the exemplary implementation, \docker\ is used as it provides an easy-to-use interface and is available for x86, as well as for ARM-based platforms.

\docker\ is a holistic containerization platform which includes everything needed to build, run, deploy, and distribute containers. The \docker\ application is structured in a client-server model (\cf \Cref{fig:docker-arch}). The server is a daemon process running at all times on the host machine. Its primary purpose is to manage containers, images, networks and data volumes, and to provide the means to  build images and run containers. The \docker\ server exposes a RESTful HTTP API by which it can be controlled. The client side of the \docker\ application is a command line interface (CLI) which acts as a user friendly façade for the server's API.\footnote{\url{docs.docker.com/engine/docker-overview}}

\begin{figure}[htpb]
  \centering
  \includegraphics[width=0.9\textwidth]{figures/docker-arch.pdf}
  \caption[\docker\ architecture]{\docker\ architecture}\label{fig:docker-arch}
\end{figure}

\pagebreak
\subsection{\docker\ Components}
\paragraph{\docker\ Images.}
Images are a central component of \docker . They may be seen as the ``blueprint'' on the basis of which containers are built and define exactly which files and directories are contained in a container once it comes to life. Effectively, every container is an instantiation of an image of a certain type, in the same way an object is an instantiation of a class in an object-oriented programming language. While the concept of images is very common among containerization and virtualization tools, \docker\ follows a rather unconventional approach.
In \docker , images are made up of series of read-only file system layers stacked on top of each other. Each layer may add files and directories to its respective underlying layer. If a layer tries to modify a file from an underlying layer it creates a copy of that file and adds it to its own layer. The modification is then performed on the copy instead of the original. This principle called ``Copy on Write'' (CoW) ensures that layers cannot be modified which makes it possible to share individual layers with other images stored on the host. An example of this is later described in \Cref{sec:containerized-services}. By treating images as a collection of individual, sharable pieces instead of atomic units, massive storage space savings can be achieved. Another benefit of this is that whenever a container is started, only a thin writable layer needs to be added on top of an image, and the underlying layers do not need to be copied. That way, start-up times of containers are kept extremely low.\footnote{\url{docs.docker.com/storage/storagedriver}} Virtual Machines, in contrast, would first need to boot into an operating system prior to operation.

\paragraph{Dockerfiles.}
\docker\ images are defined in so-called Dockerfiles.\footnote{\url{docs.docker.com/engine/reference/builder}} Dockerfiles are made up of series of instructions that tell \docker\ how to build a certain image. Each instruction adds another file system layer to the image.
\begin{lstlisting}[caption=An examplary Dockerfile, label=lst:dockerfile, numbers=left, numberstyle=\tiny]
FROM debian
COPY . /application
RUN cd /application && make
ENTRYPOINT /application/run.sh
\end{lstlisting}
An example of a Dockerfile is depicted in \Cref{lst:dockerfile}. The first line of this Dockerfile defines the base image on top of which this particular image shall be built. In this case, an image called ``debian'' was chosen. As the name suggests, that base image contains a basic installation of the Linux distribution Debian.\footnote{\url{www.debian.org}} Next, the line \ \mbox{\texttt{COPY . /application}} \ instructs \docker\ to copy the contents of the current (host) directory to the container's \ \mbox{\texttt{/application}} \ directory. Due to \docker 's CoW approach, the changes in the file system are added in the form of a layer---the underlying layers are not affected. The \ \texttt{RUN} \ instruction in the following line causes \docker\ to run the subsequent commands within the container. In this case, the newly added application directory is entered and the command \ \texttt{make} \ is executed, effectively compiling the application. Again, the resulting changes to the file system are encapsulated within another layer. In practice, the now topmost layer would contain all files produced by \ \texttt{make}. Finally, the \ \mbox{\texttt{ENTRYPOINT}} \ instruction defines the action to be performed once the container is started. In this case, the shell script \ \texttt{run.sh} \ would be executed within the container. This command, too, would add a layer to the image, albeit an empty one. By defining images as sequences of instructions, reproducibility is guaranteed.
By running the command \ \mbox{\texttt{docker build -t IMAGENAME .}} \ in the directory of the Dockerfile, an image based on that Dockerfile would be built. Using the command \ \mbox{\texttt{docker run IMAGENAME}}, a new container of that image type would be launched.

\paragraph{\docker\ Registries.}
\docker\ images may be stored in \docker\ Registries,\footnote{\url{docs.docker.com/registry}} where they can be browsed, managed, and distributed. The command \ \mbox{\texttt{docker push IMAGENAME[:TAG]}} \ causes the image with the name \ \mbox{\texttt{IMAGENAME}} \ to be uploaded to a registry. Analogously, to download an image from a registry to the local computer, one would have to run the command \ \mbox{\texttt{docker pull IMAGENAME[:TAG]}}. \docker\ registries are implemented as a server software which exposes an HTTP API (\cf \Cref{fig:docker-arch}). The registry server is open source\footnote{\url{www.github.com/docker/distribution}} and may be deployed on any server that supports it. The \docker\ company itself hosts one of such registries, Docker Hub,\footnote{\url{hub.docker.com}} which is provided as a publicly available  and free-to-use service.

\subsection{Multi-Platform Compatibility} \label{sec:multiplat}
The primary purpose of containerization is to provide portable execution environments that allow software to run on a broad range of computing systems. A limitation of containers is that they only provide portability across \emph{operating systems}, and not \emph{hardware platforms}. In other words, containers do not provide binary compatibility. The reason for this is that containerized applications run directly on the kernel of the host system, and do not build on top of a virtualization layer as classic VMs do. As a result, a containerized binary built for an x86-based processor will not run on an ARM system, and vice versa. This is problematic especially in the context of embedded systems which often rely on particular hardware architectures that favor energy efficiency over performance. Computing nodes in a data centers, on the other hand, are typically based on architectures aimed at providing a maximum level of performance. As the presented approach intends containers to be deployed on both, vehicular on-board systems and in the cloud, this poses a challenge. Two approaches exist to tackle this problem.

\subsubsection{Universal, QEMU-enabled Images}
The first approach relies on a thin platform compatibility layer that is put into the base of each container. This approach was popularized by resin.io, a company that specializes in containerization for IoT devices. On their Docker Hub page\footnote{\url{hub.docker.com/u/resin}} they provide \docker\ images which have the QEMU\footnote{\url{www.qemu.org}} machine emulator built in. Facilitated by QEMU's user emulation mode, binaries built for a given processor architecture may be executed on otherwise incompatible processors. QEMU achieves this by translating any guest system calls into host system calls. The previously mentioned images are built in a way that allows any arbitrary binary executed within such container to be run in the context of QEMU. Thus, the container may run on a multitude of hardware platforms.
This approach simplifies the software build process tremendously as only one universal container per service needs to be built. This container can then be reused for all platforms.

\subsubsection{Manifest Lookup}
The QEMU approach has a significant drawback: multicast is not well supported by QEMU, which is why the idea was discarded.
Thus, another approach was leveraged to solve the multi-platform compatibility issue. This approach builds on Continuous Integration (CI) in accord with Docker Hub, and in particular, Docker Hub's \emph{image manifest} feature. The concept of image manifests builds on the following premise: by default, a \docker\ image name corresponds to exactly \emph{one} image. Thus, whenever an image with a given name is requested from the \docker\ registry, only that specific image is returned to the requester. Image manifests extend registries by the option to define \emph{multiple} images per image name. That way, several containers, each specifically built for a given platform, may be deployed under the same name. Depending on the requesting node's hardware architecture, a different image may be returned. \Cref{fig:manifest-pull} depicts an example for this. In the example, an x86-based machine requests the \ \mbox{\texttt{some/image:v1.0}} \ image from the registry. What is then returned from the registry is an image specifically made for x86 architectures. On the other hand, an ARM-based machine which requests the very same image, \ \mbox{\texttt{some/image:v1.0}}, receives an entirely different image as a response. This kind of behavior is achieved through an image lookup in the manifest file which is performed behind the scenes.

\begin{figure}[htpb]
  \centering
  \includegraphics[width=0.8\textwidth]{figures/manifest-pull}
  \caption[Pulling \docker\ images via manifest file]{Two nodes request the same image name but receive different images which are specifically built for their respective processor architecture. Which node requires which image is determined by a look-up in the manifest file.}\label{fig:manifest-pull}
\end{figure}

A disadvantage of this approach (compared to the QEMU one) is that multiple versions of the same container need to be built every time a service receives a software update. To cope with this hindrance, a minimal CI pipeline was leveraged. Consider \Cref{fig:manifest-push} in which the employed service deployment process is depicted. The build process is initiated whenever a code change is pushed to the remote git repository. Once a push is registered, the CI Platform (Travis CI\footnote{\url{www.travis-ci.org}} was used) pulls the latest version from the git repository and executes a build script. The build script compiles several version of the service for each target platform, embeds the binaries in images and pushes those images to the \docker\ registry (Docker Hub). Thus, several images, each tailored to a different platform, may be built and deployed in one go.

\begin{figure}[htpb]
  \centering
  \includegraphics[width=\textwidth]{figures/manifest-push}
  \caption[Pushing \docker\ images via CI]{A git push to a service's git repository triggers the build process in the CI platform: the code is compiled for several platforms and the resulting images are pushed to the \docker\ registry.}\label{fig:manifest-push}
\end{figure}

%
%
%
%
%
%
%
%
%
%
%
\subsection{Containerized Services} \label{sec:containerized-services}
In the envisaged system, each service is packed into its own container with all its dependencies. That way, services may run wherever they are placed, and thus, a great deal of portability is achieved.

\Cref{fig:service-containers} shows an example of three containerized services running on two different machines, \experim{Host I} and \experim{Host II}. On the former, two instances of \experim{Service A}, and one instance of \experim{Service B} are placed. On the second host only one service instance is deployed, namely one of type \experim{Service C}. All services are packaged in their own, separate container. The containers are made up of three stacked images---the bottom two are common to all services. The bottommost image (``\emph{base image}'') contains the file system layers of a minimal operating system. In this thesis, the Linux distribution Debian was used. The base image only contains a limited selection of Linux tools, such as \ \texttt{ls}, \texttt{cat}, \texttt{ps} \ etc. The purpose of a base image is to lay a solid foundation that enables users to work comfortably within the container. In the case of Debian, the \emph{aptitude} package manager is additionally included which enables users to easily install further tools.

\begin{figure}[htpb]
  \centering
  \includegraphics[width=\textwidth]{figures/docker-sharing}
  \caption[An example of containerized services]{An example of four containerized service instances, some sharing common logic to save disk space and facilitate fast updates}\label{fig:service-containers}
\end{figure}

Next, an \emph{intermediate image} is built on top of the base image. The intermediate image adds further layers containing the services' run-time environment, and in particular, the shared libraries of OpenDDS. This image could optionally contain additional libraries and tools that are common to all services. For the purpose of demonstration, however, DDS on its own is sufficient. The base image and the intermediate image lay the foundation of all services. Any image built on top of these could run any DDS-enabled application. Facilitated by \docker 's layered approach to images, the bottom two images can be shared among all services deployed on a host. In the example, the base image and the intermediate image both only need to be stored once on \experim{Host I}, which saves tremendous amounts of disk space.

At the top, finally, sit the \emph{service images}. In these images contained is the actual service logic, and more precisely, the service's binary and configuration files. The service images are unique to each service, and are not shared, except when the same service is instantiated multiple times on the same machine (\cf \experim{Service A} in \Cref{fig:service-containers}).



\subsection{Container Networking with \wnet}

\wnet\ is an open source project\footnote{\url{www.github.com/weaveworks/weave}} which aims to provide advanced overlay networking capabilities for \docker\ containers. It is designed to make up for the shortcomings of \docker 's built-in overlay networks, and more specifically, their lack of encryption and multicast support. By means of \docker\ overlay networks, containers dispersed among physical hosts that reside in their own physical networks, may exchange data freely. \wnet\ presents the network to the applications in a location transparent way, such that from the applications' viewpoint, it does not matter whether its peers are located on the same host or within a data center on the other side of the world. \wnet\ is developed by a global team, mostly employed by the London-based software company \emph{Weaveworks}\footnote{\url{www.weave.works}}.


\paragraph{Functioning.}
\wnet\ is implemented as client software which needs to be installed on each machine that is supposed partake in the overlay. The software can be started using a single command, and in the following, all containers launched on that host will automatically connected to the overlay network. This is achieved by means of a \emph{\docker\ API proxy}. The proxy sits between \docker 's command-line client and the \docker\ daemon and intercepts all communication between the two. When the \docker\ engine is instructed to start a container, the proxy takes all precautions needed to enable overlay networking for that container. Once the connection is established, all container traffic is routed through three dedicated network channels: one TCP connection to exchange meta data about the network, and two UDP channels for duplex data exchange.


\begin{figure}[htpb]
  \centering
  \includegraphics[width=\textwidth]{figures/sdn.pdf}
  \caption[An example of containers connected via \wnet\ overlay network]{A number of dispersed containers connected by a \wnet\ overlay network}\label{fig:weavescheme} \todo[inline]{Add legend for boxes, add another host}
\end{figure}

When the Weave software is started, a central component of \wnet , the \emph{Weave router} is launched (\cf\ \autoref{fig:weavescheme}). Similar to a hardware router, the Weave router is responsible for the forwarding and routing of data packets to their appropriate receivers. Weave routers can be seen as gateways through which all containers participating in a Weave network are connected. To facilitate routing on the data plane, a custom UDP encapsulation protocol, called \emph{sleeve}, was devised. A Weave router in itself is an containerized application running at all times, in the same way a daemon would. There is one of such router containers running on every host in a Weave-enabled infrastructure.

The Weave router is a user space process. As such, a context switch is needed every time it is tasked to process a packet. This comes with a substantial performance overhead. Hence, as a faster alternative, the so-called \emph{fast datapath} mode was added. In this mode, packets are processed by the Linux kernel instead of by the Weave router. This way, the context switch into user space is omitted. \wnet\ leverages the Linux kernel's \emph{Open vSwitch datapath} module \cite{pfaff2015design} to achieve this behavior. Open vSwitch can be used to create a software-based, virtual network switch. That way, the kernel can be instructed to process packets in a certain way. For instance, the kernel can be commanded to add a VXLAN header to each packet, thereby achieving the same result as the Weave router, but faster. However, fast datapath mode can only be used when the underlying infrastructure allows it. The Internet is a particular example of a network where fast datapath communication is hard to achieve.


\paragraph{Topology Management.} 
Weave's topology management is self-governing and self-healing. Peers continually exchange topology information and monitor the state of the network. Whenever peers lose connectivity, they continuously try to re-establish the connection until it is restored. All participating peers know the topology of the entire network. For this, \wnet\ employs a sophisticated discovery and topology management mechanism by which changes in the network topology are rapidly propagated within the network. The topology management protocol is based on a spanning-tree broadcast mechanism known from hardware switches. To further ensure that all peers have an up-to-date neighbor list at all times, \wnet\ additionally employs a custom neighbor gossiping protocol by which each peer sends update messages to a random subset of their neighbors. Updates of the network's topology are performed periodically as well as when certain events occur, such as when a node joins or leaves the network.


\paragraph{Encryption.} 
As mentioned earlier, a salient feature of \wnet\ is the ability to encrypt all traffic within the overlay network. This is especially important for networks that span over insecure infrastructures, such as the Internet. To set up encryption, a shared secret (password) needs to be provided when the Weave router is launched. The shared secret must therefore be present on all participating hosts. From the password, salted, ephemeral session keys are generated which are then used to encrypt the packets. For each connection between any two peers, one unique session key exists.

The way encryption is applied differs depending on the forwarding mode used (sleeve or fast datapath). In fast datapath mode, a IPsec-based protocol is used, whereby each packet is wrapped in an encapsulated security payload (ESP). Because each packet in this mode is processed by the Linux kernel, encryption is applied by means of the standard Linux Kernel Crypto API which is thoroughly tested, and generally considered secure.

For sleeve mode, a custom encryption algorithm based on TLS\footnote{Transport Layer Security} is used. As with fast datapath encryption, sleeve mode encryption utilizes shared, ephemeral session keys for each connection. \autoref{fig:weaveencryption} depicts how these session keys are generated. In the image, two hosts ($H_1$, $H_2$) are to establish an encrypted connection. First, $H_1$ initiates the key exchange by sending a handshake message to $H_2$ \circled{1}. Then, both hosts generate their own, individual key pairs so that each host has a public key and a private key. The key pair for $H_1$ is $(K_{1P}, K_{1S})$ and the key pair for $H_2$ is $(K_{2P}, K_{2S})$. Once that is done, both hosts exchange their respective public keys, $K_{1P}$, and $K_{2P}$ \circled{3}. Using the peer's public key and their own private key, both hosts derive an auxiliary shared key, $S_A$, by means of Diffie--Hellman key exchange \cite{bresson2001provably}: $D(K_{1S},K_{2P})$ \circled{4}. Finally, the actual shared key can be generated. For this, the shared password is appended to $S_A$ to provide authenticity. In order to bring the key to the desired length of 256 bit, the compounded key is additionally hashed via SHA256: $H(S_A, Q)$ \circled{5}. The end result of this procedure is the final shared key, $S_{12}$, which is then used to encrypt the traffic between the two hosts.

\begin{figure}[htpb]
  \centering
  \includegraphics[width=0.8\textwidth]{figures/weave-encryption.pdf}
  \caption[Weave's key exchange protocol]{\wnet 's key exchange protocol in sleeve mode}\todo[inline]{remove Q arrows, make circles white, S\_12 -> S}\label{fig:weaveencryption}
\end{figure}


\todo{move or delete}Drawbacks: \wnet\ does not support IPv6 which is to the detriment of IoT networks, which benefit from the vast number of addresses of IPv6. However, in the vehicle use case, it is rather unlikely that every ECU has its own IP address. Rather, there is one gateway connected to a 5G module and all traffic is routed through that gateway. Because of this, the address scarcity issue only peripherally applies to the intended use case.








