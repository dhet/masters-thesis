
\chapter{Introduction}\label{chapter:introduction}


\section{Motivation}
Recent trends in the automotive industry are introducing new, increasingly complex software functions into vehicles \cite{broy2006challenges}. In particular, autonomous driving and advanced driver-assistance systems (ADAS) are a major force behind this development. Such systems leverage modern methods coming from artificial intelligence (AI) research which are demanding, both in terms of computational power as well as in the volume of data they require. At the same time, embedded on-board computers are typically severely limited in their computational capacities. This problem may be overcome with the aid of another trend in automotive systems: cloud connectivity. Cloud infrastructures, in essence, are remote data centers which have seemingly infinite amounts of computing resources and storage space available to them. In the future, computationally expensive functions enabling advanced functionalities, such as autonomous driving, may be offloaded from the vehicle's on-board system to the cloud. Since network connectivity is not always guaranteed, e.g. when navigating through geographically remote areas, such functions still need to work "offline". Hence, a method is needed to allow the same functions to run on resource constrained embedded systems as well as on servers in cloud infrastructures. Virtualization and containerization technologies are promising contenders to make such an endeavor feasible. By wrapping software artifacts in portable, self-contained execution environments, such tools allow applications to be deployed on a variety of operating systems and hardware platforms without requiring modifications in the application code. 

A challenge is the dynamic nature of vehicles. A vehicle may be connected to a remote server for days, and then disconnect in the next second. In such cases, the on-board system must immediately take over operation in order to stay functional and guarantee the safety of the passengers. For this, the way components communicate must be absolutely seamless, regardless of their location.

%
%
%
%
%
%
%
%
%
%

\section{Context}

\subsection{Modern Software Architectures}
Today, there is a tendency for software architectures to shift away from the \emph{monolithic} paradigm of having a single software entity deployed on a single, potent server to a more \emph{distributed} paradigm where multiple components are spread over a number of less potent physical hosts connected over a network.
The benefits of this approach became evident in the early 2000's when a new type of software architecture utilizing this concept became popular: \emph{service-oriented architectures} (SOAs).

In recent years, SOAs have made a revival in the form of \emph{microservices}, which, in essence, is a more fine grained type of SOA that avoids the complexity induced by \emph{Enterprise-Service-Buses} (ESBs). Similar to SOA, the idea behind microservices is to split functionality into a collection of reusable components, each fulfilling a single specific purpose. Recent research efforts are evaluating the possibility to include such paradigm in the realm of automotive software systems -- in part achieving great success \cite{berger2017containerized}. 

Since Microservices are very limited in scope they may be built and deployed in a matter of minutes, rather than hours, as is the case with monolithic applications. Thus, microservices have the potential to tremendously accelerate the software development process when being combined with modern development practices such as \emph{continuous integration / continous delivery} (CI/CD).


%
%
%
%
%
%
%
%
%
%

\section{Objective}

The aim of this thesis is to research the possibility of enabling cloud connectivity in future automotive software architectures.
 
Furthermore, containerization technologies are evaluated regarding their aptitude in safety-critical systems. Moreover, a technical realization of a repartitioning system is presented which is deployed on a specifically-built testbed and then evaluated regarding performance and safety.


Goals:
\begin{itemize}
\item Offload computations to the cloud
\item Run the same functions within vehicle and in the cloud
\item Must deal with spotty reception
\item Therefore: Seamless switch between cloud and on-board
\item System needs to be modular to allow for fine-grained control over which functionality is offloaded 
\item provide high degree of distribution transparency
\end{itemize}


%
%
%
%
%
%
%
%
%
%


\section{Structure of the Thesis}

The remainder of this work is structured as follows. First, preliminaries are given in \autoref{chapter:preliminaries}, briefly explaining all concepts and technologies relevant to the presented approach. The next chapter, \autoref{chapter:related-work}, is dedicated to the summarization and discussion of similar approaches and other related work. In the subsequent chapter, \autoref{chapter:realization}, the actual realization of the approach is described, going into detail about how technologies were leveraged to implement a cloud-enabled automotive software architecture. In \autoref{chapter:evaluation}, the approach is evaluated. The evaluation is supported by a number of benchmarks which were conducted to assess the approaches feasibility and to quantify its quality attributes. What then follows in \autoref{chapter:discussion} is a discussion about the approaches aptitude as well as its limitations. The thesis concludes with \autoref{chapter:conclusion} where this work is summarized and future work is suggested.

%
%
%
%
%
%
%
%
%
%
%
%
%
%
%
%
%
%
%
%
%
%
%
%
%
%
%
%
%
%
%
%
%
%
%
%
%
%
