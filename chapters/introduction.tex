
\chapter{Introduction}\label{chapter:introduction}

\section{Motivation}
Recent trends in the automotive industry are introducing new, increasingly complex software functions into vehicles \cite{broy2006challenges}. In particular, autonomous driving and advanced driver-assistance systems (ADAS) are a major force behind this development. Such systems leverage modern methods coming from computer vision and artificial intelligence (AI) research which are demanding, both in terms of computational power as well as in the volume of data they require. For instance, ``Junior'', Stanford’s
entry in the 2007 DARPA Urban Challenge,\footnote{The DARPA Urban Challenge was a competition for autonomous vehicles.} uses a planning system which generates thousands of trajectories per second and then chooses the one that best fits the current situation~\cite{levinson2011towards}. Another example is ``Boss'', the winner of the challenge, which utilizes a similar planning algorithm that, under consideration of the vehicle's state, generates sequences of feasible maneuvers to dodge obstacles on the road~\cite{urmson2009autonomous}. These are just two examples of computationally demanding processes which will become commonplace with the continuous progression towards fully automated driving.
The increased computational demand is in conflict with current embedded on-board computers which are typically severely limited in their computational capacities. As these fail to address the increased needs, additional, more powerful computing nodes are being implemented in automated driving test vehicles (\eg\ \cite{levinson2011towards, urmson2009autonomous, rauskolb2008caroline}). This solution is less than optimal as the added hardware is expensive, takes up space and increases the weight of the vehicles. Furthermore, energy consumption will become an increasingly important factor to consider as electric mobility will play a major role in the future \cite{festag2016studie}. The added hardware will unnecessarily draw power which is predominantly intended to fuel the vehicle itself. Thus, while additional on-board computing hardware is inevitable, a goal should be to reduce it by as much as possible~\cite{white2010r}.

A potential remedy to overcome this problem is cloud computing \cite{mell2011nist}. Cloud infrastructures, in essence, are remote data centers which have seemingly infinite amounts of computing resources and storage space available to them. Through auto-scaling, resources can be solicited according to situational demand which prevents overprovisioning and helps to reduce operational costs and to save energy. In the future, computationally intensive functions enabling advanced functionalities, such as autonomous driving, may be supported by---or even entirely offloaded to---the cloud \cite{liu2017unified}. This endeavor will be greatly facilitated by the upcoming 5G cellular network which enables high throughput data transmission while providing low latencies and reduced power consumption \cite{andrews2014will}.

Despite the fact that considerable efforts are currently made to push the wide spread adoption of 5G \cite{festag2016studie}, it can not realistically be guaranteed that vehicles will be connected to the Internet at all times, \eg , when navigating through geographically remote areas. In addition to the deficient availability of cellular networks, connectivity in mobile systems is characterized by frequent successions of connection losses and reconnections. For this reason it is vital that all critical vehicular functions remain ``offline'', \ie , the vehicle's operability should not be reliant on cloud services. Rather, the cloud should fulfill a \emph{supportive} role.
%Hence, a method is needed to allow the same functions to run on resource constrained embedded systems as well as on high performance servers in the cloud \cite{white2010r}. Virtualization and containerization technologies are promising contenders to make such an endeavor feasible.

%
%
%
%
%
%
%
%
%
%

\section{Objective}
The goal of this thesis is to devise a system that connects vehicles to the cloud as means to
\begin{itemize}
\item offload computations and reduce energy consumption,
\item facilitate data collection for real-time monitoring and analytics purposes, and to
\item introduce redundancy in an effort to improve road traffic safety.
\end{itemize}
The system ought to be unobtrusive in that the cloud shall not interfere with the operation of the vehicle itself. Much rather, the cloud shall act as an optional addendum that performs supportive tasks on the behalf of the vehicle. 
%To be able to control exactly which functionality shall be offloaded it is necessary to separate the vehicle's functionality into fine-grained tasks. To this end, the system shall build on a modular design which allows for the segmentation of a vehicle's functionality into single, isolated units. The separation mechanism shall present a view on the system as a collection of functional units, of which each one can be moved between vehicle and cloud independently. A requirement for this is that the same function may run on a vehicle's on-board system as well as on a cloud-provisioned machine in the exact same way. A solution to this problem is to be provided. Furthermore, each functional unit shall be able to communicate with other functional units, regardless of their physical locality. \Ie , it shall not matter whether two interacting components are located within the vehicle or whether one of them is deployed in the cloud. This serves to transform today's \emph{fixed} vehicle boundaries into \emph{elastic} boundaries which may shape depending on the current situation's requirements.
%The system shall furthermore support redundancy: it shall be possible to have several versions of a vehicular function running simultaneously to enable fail-safe behavior. To guarantee consistency between the redundant components, they all need to be updated simultaneously. This calls for advanced messaging techniques that support the transmission of data to multiple receivers.
%For instance, take a function $F_\alpha$ which continually transmits sensor data to another function, $F_\beta$, which processes that data. A replica of function $F_\beta$ runs in the cloud as a backup measure. The data emitting function ($F_\alpha$) needs to be able to send the same data samples to both receivers simultaneously.
%In this sense, the system shall act similar to a message bus. Unlike traditional buses however, the envisaged bus shall extend into the cloud in a location-transparent fashion.
A major challenge in this enterprise is the mobility aspect of cars. It cannot always be guaranteed that a connection to the cloud is available. As a consequence, cloud-based functions may be available for a few seconds, and then become unavailable again. The system must be able to deal with extreme fluctuations in the network's availability. When connectivity is restored, operation must resume as if no interruption occurred. A seamless and fast transition from cloud-operation to on-board-operation and back again must be possible.

\paragraph{Disclaimer.}
It must be stressed that it is \emph{not} the goal of this thesis to realize a fully functional, ready-to-use system, but merely to present a working proof of concept. For the implementation, concessions \wrt\ safety and other requirements relevant to automotive use cases will have to be made and questions regarding the system's practical usability will remain unanswered. This thesis presents a technical realization which allows for the migration of vehicular functionality to the cloud. As such, it is only concerned with architectural and implementation aspects.
It is not the purpose to present high-level repartitioning strategies or a management and orchestration system. Such topics may build \emph{on top} of this thesis' approach.
%
%
%
%
%
%
%
%
%
%
\section{Contributions}
In this thesis, a novel method to address the previously stated problem is presented. The prototypical implementation features a combination of technologies which, to the best of the author's knowledge, has not been subject to scientific investigations yet. Thus, in this work, experiments are presented on the basis of which the interplay of these technologies is evaluated.
Moreover, in the process of writing this thesis, contributions in the form of two research papers were made.
Firstly, the paper titled ``Data-Centric Communication and Containerization for Future Automotive Software Architectures'' \cite{kugele:hettler:peter:icsa18} investigates the prospect of leveraging publish-subscribe communication and containerization technology as building blocks to devise an automotive service-oriented architecture (SOA). This paper was presented and published at the IEEE International Conference on Software Architectures 2018 (ICSA) in Seattle.
Secondly, a paper titled ``Elastic Service Provision for Intelligent Vehicle Functions'' \cite{kugele:hettler:itsc} was created in which the main contribution of this thesis was used as foundation to concept an intelligent management component that allows for vehicular E/E systems to dynamically adapt to operational situations. At the time of this writing, this paper is yet to be published.
%
%
%
%
%
%
%
%
%
%
\section{Structure of the Thesis}
The remainder of this work is structured as follows. First, preliminaries are given in \Cref{chapter:preliminaries}, briefly explaining all concepts and technologies relevant to the presented approach. The next chapter, \Cref{chapter:related-work}, is dedicated to the summarization and discussion of similar approaches and other related work. Chapter~\ref{chapter:approach} provides a conceptual description of the approach and lists example use cases and requirements. In the subsequent chapter, \Cref{chapter:realization}, the actual realization of the approach is described, going into detail about how technologies were leveraged to exemplarily implement the idea. In \Cref{chapter:evaluation}, the approach is evaluated. The evaluation is supported by a number of benchmarks which were conducted to assess the system's feasibility and to quantify its quality attributes. What then follows in \Cref{chapter:discussion} is a discussion about the system's aptitude as well as its limitations. The thesis concludes with \Cref{chapter:conclusion} where this work is summarized and future work is suggested.

%
%
%
%
%
%
%
%
%
%
%
%
%
%
%
%
%
%
%
%
%
%
%
%
%
%
%
%
%
%
%
%
%
%
%
%
%
%
