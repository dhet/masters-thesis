
\chapter{Introduction}\label{chapter:introduction}


\section{Motivation}
Recent trends in the automotive industry are introducing new, increasingly complex software functions into vehicles \cite{broy2006challenges}. In particular, autonomous driving and advanced driver-assistance systems (ADAS) are a major force behind this development. Such systems leverage modern methods coming from computer vision and artificial intelligence (AI) research which are demanding, both in terms of computational power as well as in the volume of data they require. \Eg , in order for planning systems for autonomous driving to produce the best possible results, thousands of trajectories must be generated per second and then the one that fits best must be selected \cite{levinson2011towards}. At the same time, embedded on-board computers are typically severely limited in their computational capacities. For this reason, additional, more powerful computing nodes have been implemented in early vehicles. This solution is less than optimal as the added hardware is expensive, takes up space and increases the weight of the vehicles. Furthermore, energy consumption will become an increasingly important factor to consider as electric mobility will play a major role in the future \todo{citation?} . The added hardware will unnecessarily draw power which is predominantly intended to fuel the vehicle itself. Thus, while additional on-board computing hardware is inevitable, a goal should be to reduce it by as much as possible. 

A potential remedy to overcome this problem is cloud computing \cite{mell2011nist}. Cloud infrastructures, in essence, are remote data centers which have seemingly infinite amounts of computing resources and storage space available to them. In the future, computationally intensive functions enabling advanced functionalities, such as autonomous driving, may be supported by---or even entirely offloaded to---the cloud. This endeavor will be greatly facilitated by the upcoming 5G cellular network which enables high throughput data transmission while providing low latencies and reduced power consumption \cite{andrews2014will}.

Despite the fact that considerable efforts are currently made to push the wide spread adoption of 5G, it can not realistically be guaranteed that vehicles will always be connected to the Internet, \eg , when navigating through geographically remote areas. In addition to the deficient availability of cellular networks, connectivity in mobile systems is characterized by frequent successions of connection losses and reconnections. For this reason it is vital that all critical vehicular functions still work "offline". Hence, a method is needed to allow the same functions to run on resource constrained embedded systems as well as on high performance servers in the cloud. Virtualization and containerization technologies are promising contenders to make such an endeavor feasible. By wrapping software artifacts and their dependencies in portable, self-contained execution environments, such tools allow applications to be deployed on a variety of operating systems and hardware platforms without requiring modifications in the application code.
\todo[inline]{abgehackt? soll virtualisierung jetzt schon besprochen werden?}

%
%
%
%
%
%
%
%
%
%

\section{Objective}
The goal of this thesis is to present a system that connects vehicles to the cloud as means to \begin{itemize}
\item offload computations and reduce energy consumption,
\item facilitate data collection for real-time monitoring and analytics purposes, and to
\item increase redundancy in an effort to improve safety.
\end{itemize} 
The system ought to be unobtrusive in that the cloud shall not interfere with the operation of the vehicle itself. Much rather, the cloud shall act as an optional addendum that performs supportive tasks on the behalf of the vehicle. A separation of the vehicle's holistic functionality into fine-grained tasks is necessary to achieve this. To this end, the system shall build on a modular design which allows for the segmentation of a vehicle's functionality into single, isolated units. The separation mechanism shall present a view on the system as a collection of functional units, of which each one can be moved between vehicle and cloud independently. A requirement for this is that the same function may run on a vehicle's on-board system as well as on a cloud-provisioned machine in the exact same way. A solution to this problem is to be provided. Furthermore, each functional unit shall be able to communicate with other functional units, regardless of their physical locality. \Ie , it shall not matter whether two interacting components are located within the vehicle or whether one of them is deployed in the cloud. This serves to transform today's \emph{fixed} vehicle boundaries into \emph{elastic} boundaries which may shape depending on the current situation's requirements.

The system shall furthermore support redundancy: it shall be possible to have several versions of a vehicular function running simultaneously to enable fail-safe behavior. To guarantee consistency between the many redundant components, they all need to be updated simultaneously. This calls for advanced messaging techniques that support the transmission of data to multiple receivers. For instance, take a function $F_\alpha$ which continually transmits sensor data to another function, $F_\beta$, which processes that data. A replica of function $F_\beta$ runs in the cloud as a backup measure. The data emitting function ($F_\alpha$) needs to be able to send the same data samples to both receivers simultaneously. In this sense, the system shall act similar to a message bus. Unlike traditional buses however, the envisioned bus shall extend into the cloud in a location-transparent fashion.

A major challenge in this enterprise is the mobility aspect of cars. It cannot always be guaranteed that a connection to the cloud is available. As a consequence, functions may be available for a few seconds, and then become unavailable again. The system must be able to deal with extreme fluctuations in the network's availability. When connectivity is restored, operation must resume as if no interruption occurred. A seamless transition from cloud-operation to on-board-operation and back again must be possible.


\paragraph{}

\todo[inline]{non goals?}
What is NOT the purpose of this thesis?
\begin{itemize}
\item It is not he goal to present a repartitioning strategy for the offloading of computations but to present a way to on the basis of which such systems may be implemented.
\end{itemize}


%
%
%
%
%
%
%
%
%
%


\section{Structure of the Thesis}
The remainder of this work is structured as follows. First, preliminaries are given in \autoref{chapter:preliminaries}, briefly explaining all concepts and technologies relevant to the presented approach. The next chapter, \autoref{chapter:related-work}, is dedicated to the summarization and discussion of similar approaches and other related work. In the subsequent chapter, \autoref{chapter:realization}, the actual realization of the approach is described, going into detail about how technologies were leveraged to exemplarily implement the idea. In \autoref{chapter:evaluation}, the approach is evaluated. The evaluation is supported by a number of benchmarks which were conducted to assess the system's feasibility and to quantify its quality attributes. What then follows in \autoref{chapter:discussion} is a discussion about the system's aptitude as well as its limitations. The thesis concludes with \autoref{chapter:conclusion} where this work is summarized and future work is suggested.

%
%
%
%
%
%
%
%
%
%
%
%
%
%
%
%
%
%
%
%
%
%
%
%
%
%
%
%
%
%
%
%
%
%
%
%
%
%
