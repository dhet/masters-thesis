\chapter{Related Work}\label{chapter:related-work}

\paragraph{Similar Approaches}

In their paper, \citeauthor*{berger2017containerized} \cite{berger2017containerized} describe their experiences of employing containerization in the context of self-driving vehicles. 
In accordance with the microservice architecture pattern, vehicular functionality in their approach is split into a number of fine-granular services deployed on distributed nodes within the vehicle. The services are connected via \emph{OpenDaVINCI}\footnote{code.opendavinci.org}, a real-time capable middleware for publish-subscribe communication. This approach is very similar to the one proposed in this thesis in that communication is based on multicast over a middleware. However, their approach lacks cloud connectivity. The authors furthermore struggle to get multicast to work over Docker overlay networks -- a problem that was solved in this thesis. 

%The interplay of their services is based on a \emph{pipes-and-filters} methodology.
%Multi platform images are achieved through different base layers employing different compilers. Disadvantage: containers cannot be moved freely between nodes of different hardware architectures.
%questionable design decisions: Centralized configuration protocol... but decentralization is key
%Suggest to have private registry within the vehicle.
%To provide a high degree of tracability, they present a versioning system that allows versioning of individual image layers which seamlessly integrates into their development workflow.

%
%
%
%
%

\citeauthor*{schneider2016achieving} \cite{schneider2016achieving} also use containerization in a microservice environment for vehicular functionality. However, the authors only focus on the backend part, \ie , services provisioned in data centers that cars may connect to. Their approach does not consider the use of microservices within the vehicle and is generally described in rather vague terms.

%
%
%

\paragraph{DDS over SDN}

\citeauthor*{farahzadi2017middleware} \cite{farahzadi2017middleware} analyze the problem domain of cloud connectivity for the IoT. For this purpose, they present a survey in which they analyze a number of different IoT middleware solutions and compile a set of key challenges and requirements for such systems.

\citeauthor*{hakiri2015publish} \cite{hakiri2015publish} devise a system to interconnect IoT devices via DDS over an overlay network. However, the description of their approach lacks details and no real effort were made to evaluate it.


\paragraph{Docker specific}

Relevant for the evaluation part of this thesis (\autoref{chapter:evaluation}) is \citeauthor*{kratzke2017microservices}'s paper \cite{kratzke2017microservices} on the performance impact of using encrypted overlay networks for Linux containers on top of hypervisors. This setup -- containers atop hypervisors -- is very common in cloud scenarios as IaaS providers often provision servers in the form of virtual machines. This is also the case for the test setup described in this thesis. The author concludes that containers add a non-negligible impact on networking performance. In Weave Net-based overlay networks, the performance is further impaired. A minimal impact was observed when using encryption.

% does not consider weave fast datapath
% argues that weave would be faster if the weave routers weren't packaged in containers


\paragraph{DDS + Virtualization}
\cite{serrano2013virtualizing}: Virtualizing DDS middleware: performance challenges and measurements

\citeauthor*{perez:gutierrez:ieeetpds16} \cite{perez:gutierrez:ieeetpds16} present a way to combine DDS and hypervisor-based inter-process communication to interconnect a time and space partitioned system. Their focus lies on mixed-criticality applications running atop a hypervisor designed for safety-critical scenarios. The performance tests they conduct show that DDS adds a reasonable overhead to the communication performance but the overhead caused by hypervisor-based partitioning is quite significant, albeit reducible by the use of multicore processors.

\cite{garcia2013benchmarking}: Benchmarking communication middleware for cloud computing virtualizers. Test performance overhead of hypervisors for DDS as well as the performance overhead of DDS compared to ICMP packets

\paragraph{Publish-Subscribe in Embedded Systems}

\cite{bouhouch2013dds}: DDS on top of FlexRay

\cite{sommer2013race}: RACE: A centralized Platform computer based architecture for automotive applications


Adaptive AUTOSAR's underlying messaging middleware is \emph{SOME/IP}

Can be used on top of AUTOSAR.

Service discovery based on modes. Client: request / listen, server: offer / silent.
Service discovery takes is pretty fast < 10 ms. Depends on parameters.

Designed to be used over Ethernet.

Network representation and internal representation are similar to allow for fast serialization at the expense of message size. But this is not a problem since the transport channel doesn't have stringent bandwidth and frame size limitations.

Supports TCP and therefore bandwidth intensive streams.

Supports several communication models: request/response, async, events, "field", event groups (PubSub)


\paragraph{Cloud}

\cite{greenberg2008cost}: The cost of a cloud: research problems in data center networks

Insights: location-independent addressing is a major design objective

\cite{wang2010impact}: The impact of virtualiztion on network performance.

The authors perform networking benchmarks on Amazon EC2 instances and find that virtualization and processor sharing heavily impacts networking performance. In particular, network delay and TCP/UDP throughput suffers drastically.


\paragraph{computation offloading in cloud (cell phones)}
\citeauthor*{barbera2013offload} \cite{barbera2013offload} analyze in which cases it is sensible to offload computations from cell phones to the cloud. There is a tradeoff to be made between the energy savings achieved by the offloading itself and the increased computational overhead of sending data over a mobile network. They aim to provide virtual replicas of the hardware in the cloud, running at all times alongside the physical device to support it whenever there is a need. They do not consider statelessness of the replicas and employ a synchronization mechanism to keep the state consistent. This incurs additional overhead. They use cell phone specific tools which are inapplicable in the automotive use case. They view their approach as holistic solution -> lack of flexibility

In \cite{chun2011clonecloud} \citeauthor*{barbera2013offload} present \emph{Clone Cloud}, a system to offload workload from mobile applications to the cloud. They employ a combination of static analysis and dynamic profiling to automatically partition mobile applications. repartitioning is thread-wise. They achieve a 20-fold execution speed-up and 20-fold decrease of energy consumption.

MAUI is not as automatic. Requires program code to be annotated.


\paragraph{Embedded SOA}
Work on service-oriented architectures is available in abundance. However, few attempts were made to build such architecture on top of embedded systems

\begin{itemize}
%\item \cite {scholz2009soa}: \citeauthor*{scholz2009soa} - ∈ SOA-Service Oriented Architectures adapted for embedded networks
\item \cite{kugele2017service}: \citeauthor*{kugele2017service} - On Service-Orientation for Automotive Software
\item \cite{wagner2014developing}: \citeauthor*{wagner2014developing} - Developing self-adaptive automotive systems - On the integration of service-orientation into automotive
development processes -- SOMA4DDAS
\end{itemize}





\paragraph{Other}
\citeauthor*{broy2006challenges} \cite{broy2006challenges}: Challenges in automotive software engineering... Takeaways: increasing amounts of code in cars, increasing complexity. Software is the main innovation driver. Broy states challenges, gives an industry profile of automotive industry. Lists requirements that are very different from other domains. Current development practices in the industry are lacking.
