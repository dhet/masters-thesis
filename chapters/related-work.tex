\chapter{Related Work}\label{chapter:related-work}

In their paper, Berger et. al. \cite{berger2017containerized} describe their experiences of employing containerization in a vehicular microservice architecture. 

They highlight Docker's advantages: accelerates software testing, quickly onboard new team members

In accordance to the microservice architecture pattern, vehicular functionality in their approach is split into a number of fine-granular services, each having very little responsibilities. The services are connected via a middleware



\section{Adaptive AUTOSAR}

Adaptive AUTOSAR's underlying messaging middleware is \emph{SOME/IP}

Can be used on top of AUTOSAR.

Service discovery based on modes. Client: request / listen, server: offer / silent.
Service discovery takes is pretty fast < 10 ms. Depends on parameters.

Designed to be used over Ethernet.

Network representation and internal representation are similar to allow for fast serialization at the expense of message size. But this is not a problem since the transport channel doesn't have stringent bandwidth and frame size limitations.

Supports TCP and therefore bandwidth intensive streams.

Supports several communication models: request/response, async, events, "field", event groups (PubSub)