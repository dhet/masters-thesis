
\chapter{Conclusion}\label{chapter:conclusion}

\section{Summary}

Recent trends in automotive require innovative approaches to facilitate communication between distributed components and handle resource utilization. because traditional technologies are reaching their limits. 

Current research discusses the possibility to use publish-subscribe middleware in automotive systems. This is necessary because.... DDS is suitable for this purpose because.... 

This thesis presented a novel approach that brings together reliable communication and cloud computing. The approach utilizes a combination of different technologies to warrant for a high degree of location transparency, decoupling,  and ...

The crux of the presented approach is that services may be deployed on ECUs within the vehicle as well as within remote data centers. At the same time, communication between the services is not affected in any way. \Ie , from the viewpoint of a service deployed within the vehicle it doesn't matter whether a peer service is deployed in the same vehicle or in a remote data center -- the way they communicate is exactly the same.


Linux containers are used to package applications in a platform independent way so that they may run on on-board computers as well as in data centers. The same code is shared between both platforms.

With the aid of state-of-the-art SDN technology a VXLAN-based overlay network is created that connects containers running within the vehicle with other containers deployed in remote data centers. 


\wnet\ is not yet production ready. Unsolved problems: shared secret distribution and key renewal protocols.

The approach is by no means production ready.

\section{Contributions}

What makes the approach unique? DDS over overlay network


\section{Future Work}

The presented approach does not consider that certain functionalities have different hardware requirements, \ie , a service requiring camera footage needs to be deployed on a node that has a camera attached to it. The approach does not yet accommodate this. Container orchestration systems such as kubernetes or Docker swarm offer a solution to this problem. Such systems may control compatibilities through \emph{tags} that may be attached to containers and nodes alike. Only if a node and a container have matching tags then the container may be deployed on that node. In the example, a tag like "front-camera" could be used to indicate that a service may only be deployed on a node with an attached camera. A container orchestration system could be easily integrated into the presented approach. 

Integrate into real vehicle connected over 5G: the approach may benefit greatly from 5G's network slicing, which facilitates advanced network scaling techniques. By means of slicing, the network is separated into a number isolated slices which are then used by different applications. Each slice is highly specific to its application's use case and can enforce corresponding QoS requirements.

Maybe in the future: system inspired by network function chaining, where load is dynamically distributed on different nodes depending on a number of factors, \eg , current load, computational power, etc. Optimization problem.



Further testing of Weave: How fast can \wnet\ reconnect?

Can weave be used in PaaS systems or EC2?


The proposed system is not yet truly ubiquitous. So far, the system connects two entities: the vehicle, and the cloud. In the future, V2X will become increasingly important, especially in the light of recent autonomous driving advances. A desirable goal is to extend the system to other road-traffic participants, such as other cars, and the surrounding infrastructure. 